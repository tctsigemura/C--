% 
%  4章 ライブラリ関数
%
\chapter{ライブラリ関数}

\cmml で使用できる関数です。
必要最低限の関数が、\tac 版、\cl トランスレータ版で使用できます。

\section{標準入出力ライブラリ}

\verb/#include <stdio.hmm>/を書いた後で使用します。
トランスレータ版では\cl の高水準I/O関数の呼出しに変換されます。
\tac 版でも入出力の自動的なバッファリングを行います。
\tac 版ではバッファサイズは 128 バイトです。
以下の関数が使用できます。

\subsection{printf 関数}
標準出力ストリームに\verb/format/文字列を用いた変換付きで出力します。
出力した文字数を関数の値として返します。
トランスレータ版では\cl の\verb/printf/関数の呼出しに変換されます。
\tac 版では簡易\verb/printf/関数を使用できます。

\tac 版の簡易\verb/printf/関数では、
\verb/format/文字列に以下のような変換を記述できます。

\begin{quote}
\begin{verbatim}
%[-][数値]変換文字
\end{verbatim}
\end{quote}

\verb/-/を書くと左詰めで表示します。
数値は表示に使用するカラム数を表します。
数値を\verb/0/で開始した場合は、
数値の右づめ表示で空白の代わりに\verb/0/が用いられます。
使用できる変換文字は次の表の通りです。

\begin{quote}
\begin{tabular}{c|l}
\multicolumn{1}{c|}{変換文字} & \multicolumn{1}{c}{意味} \\\hline
\verb/o/ & 整数値を8進数で表示する \\
\verb/d/ & 整数値を10進数で表示する \\
\verb/x/ & 整数値を16進数で表示する \\
\verb/c/ & ASCIIコードに対応する文字を表示する \\
\verb/s/ & 文字列を表示する \\
\verb/%/ & \verb/%/を表示する \\
\end{tabular}
\end{quote}

\subsection{puts関数}

標準出力ストリームへ1行出力します。
エラーが発生した場合は\verb/null/を、正常時には\verb/s/を返します。

\begin{quote}
\begin{verbatim}
public char[] puts(char[] s);
\end{verbatim}
\end{quote}

\subsection{putchar関数}

標準出力ストリームへ1文字出力します。
エラーが発生した場合は\verb/true/を、正常時には\verb/false/を返します。

\begin{quote}
\begin{verbatim}
public boolean putchar(char c);
\end{verbatim}
\end{quote}

\subsection{getchar関数}

標準入力ストリームから1文字入力します。
\cl の\verb/getchar/関数と異なり\verb/char/型なので EOF チェックができません。
現在のところ、\tacos では標準入力を EOF にする方法は準備されていません。

\begin{quote}
\begin{verbatim}
public char getchar();
\end{verbatim}
\end{quote}

\subsection{fopen関数}

ファイルを開きます。
\verb/path/引数はファイルへのパス、
\verb/mode/引数はオープンのモードです。
パスは``\verb;/;''区切りで表現します。
\tacos にはカレントディレクトリはありません。
\verb/fopen/は正常に\verb/FILE/構造体、
エラー時に\verb/null/を返します。

\begin{quote}
\begin{verbatim}
public FILE fopen(char[] path, char[] mode);
\end{verbatim}
\end{quote}

\verb/mode/の意味は次の通りです。

\begin{quote}
\begin{tabular}{c|l}
\multicolumn{1}{c|}{mode} & \multicolumn{1}{c}{意味} \\\hline
\verb/r/ & 読み込みモードで開く \\
\verb/w/ & 書き込みモードで開く(ファイルが無ければ作る) \\
\verb/a/ & 追記モードで開く(ファイルが無ければ作る)
\end{tabular}
\end{quote}

\subsection{fclose関数}

ストリームをクローズします。
\tacos では、
標準入出力ストリーム(\verb/stdin/、\verb/stdout/、\verb/stderr/)を
クローズすることはできません。
\verb/fclose/は正常に\verb/false/、
エラー時に\verb/true/を返します。

\begin{quote}
\begin{verbatim}
public boolean fclose(FILE stream);
\end{verbatim}
\end{quote}

\subsection{fprintf関数}

出力ストリームを明示できる\verb/printf/関数です。
\verb/stream/引数に出力先を指定します。
出力ストリームは、\verb/fopen/で開いたファイルか\verb/stdout/です。

\begin{quote}
\begin{verbatim}
public int fprintf(FILE stream, char[] format, ...);
\end{verbatim}
\end{quote}

\subsection{fputs関数}

出力ストリームを明示できる\verb/puts/関数です。
\verb/stream/引数に出力先を指定します。
出力ストリームは、\verb/fopen/で開いたファイルか\verb/stdout/です。

\begin{quote}
\begin{verbatim}
public char[] fputs(char[] s, FILE stream);
\end{verbatim}
\end{quote}

\subsection{fputc関数}

出力ストリームを明示できる\verb/putchar/関数です。
\verb/stream/引数に出力先を指定します。
出力ストリームは、\verb/fopen/で開いたファイルか\verb/stdout/です。

\begin{quote}
\begin{verbatim}
public boolean fputc(char c);
\end{verbatim}
\end{quote}

\subsection{fgets関数}

任意の入力ストリームから1行入力します。
入力は\verb/buf/に文字列として格納します。
\verb/n/引数には\verb/buf/のサイズを渡します。
通常、\verb/buf/には\verb/\n/も格納されます。
\verb/fgets/は、EOFで\verb/null/を、
正常時には\verb/buf/を返します。

\begin{quote}
\begin{verbatim}
public char[] fgets(char[] buf, int n, FILE stream);
\end{verbatim}
\end{quote}

\cmm では、\cl の\verb/gets/が使用できません。
\verb/gets/はバッファオーバーフローの危険があるので\cmm には持込みませんでした。
\cmm で、\verb/gets/を使用したい時は\verb/fgets/を使用して次のように書きます。

\begin{quote}
\begin{verbatim}
while (fgets(buf, N, stdin)!=null) { ...
\end{verbatim}
\end{quote}


\subsection{fgetc関数}

任意の入力ストリームから1文字入力します。
\cl の\verb/fgetc/関数と異なり\verb/char/型なので EOF チェックができません。
\tac 版では安全のため、
\verb/fgetc/関数がEOFに出会うと強制的にプログラムを終了する仕様になっています。
\verb/fgetc/関数を実行する前に、必ず、
\verb/feof/関数を用いて EOF チェックをする必要があります。

\begin{quote}
\begin{verbatim}
public char fgetc(FILE stream);
\end{verbatim}
\end{quote}

\subsection{feof関数}

入力ストリームが EOF になっていると \verb/true/ を返します。
\verb/fgetc/ を実行する前に EOF チェックのために使用します。

\begin{quote}
\begin{verbatim}
public boolean feof(FILE stream);
\end{verbatim}
\end{quote}

\subsection{ferror関数}

ストリームがエラーを起こしていると \verb/true/ を返します。

\begin{quote}
\begin{verbatim}
public boolean ferror(FILE stream);
\end{verbatim}
\end{quote}

\subsection{fflush関数}

出力ストリームのバッファをフラッシュします。
入力ストリームをフラッシュすることはできません。
正常時\verb/false/、エラー時\verb/true/ を返します。
\verb/stderr/はバッファリングされていないので、
フラッシュしても何も起きません。

\begin{quote}
\begin{verbatim}
public boolean fflush(FILE stream);
\end{verbatim}
\end{quote}

\subsection{perror関数}

現在の\verb/errno/グローバル変数の値に応じたエラーメッセージを表示します。
\verb/msg/はエラーメッセージの先頭に付け加えます。
\verb/errno/はシステムコールやライブラリ関数がセットします。
\tabref{chap4:err}に\tac 版のエラーとメッセージの一覧を示します。

\begin{quote}
\begin{verbatim}
public void perror(char[] msg);
\end{verbatim}
\end{quote}

\subsection{プログラム例}

\cmml で記述した、標準入出力関数の使用例を以下に示します。

\subsubsection{TacOS 専用版のプログラム例}

\tacos では、\verb/errno/変数にセットされるエラー番号が負の値になっています。
また、アプリケーションが負の終了コードで終わった場合、
シェルがエラーメッセージを表示する仕様になっています。

更にライブラリは、ユーザプログラムのバグが原因と考えられるエラーや、
メモリ不足のような対処が難しいエラーが発生したとき、
負の終了コードでプログラムを終了します。
そこで、次のようなエラー処理を簡略化したプログラムを書くことができます。

%このプログラムでは、
%メモリ不足で\verb/FILE/構造体の割り付けができない場合は、
%\verb/fopen/内部でエラー終了しシェルがエラーメッセージを表示します。

\begin{mylist}
\begin{verbatim}
// ファイルの内容を表示するプログラム
// (TacOS 専用バージョン)
#include <stdio.hmm>
#include <errno.hmm>
public int main(int argc, char[][] argv) {
  FILE fp = fopen("a.txt", "r");
  if (fp==null) exit(errno);     // エラー表示をシェルに任せる
  while (!feof(fp))
    putchar(fgetc(fp));
  fclose(fp);
  return 0;
}
\end{verbatim}
\end{mylist}

\subsubsection{TacOS トランスレータ共通版のプログラム例}

\cl プログラム風に記述することもできます。
この例では、\verb/fopen/がエラーになったファイルの名前をエラーメッセージに
含めることができます。

\begin{mylist}
\begin{verbatim}
// ファイルの内容を表示するプログラム
// (トランスレータ、TacOS 共通バージョン)
#include <stdio.hmm>
public int main(int argc, char[][] argv) {
  char fname = "a.txt";
  FILE fp = fopen(fname, "r");
  if (fp==null) {
    perror(fname);     // エラー表示を自分で行う
    return 1;
  }
  while (!feof(fp))
    putchar(fgetc(fp));
  fclose(fp);
  return 0;
}
\end{verbatim}
\end{mylist}

\begin{mytable}{tbhp}{エラー一覧}{chap4:err}
\begin{tabular}{l|l|l}
\multicolumn{1}{c|}{記号名}
 & \multicolumn{1}{c|}{メッセージ}
 & \multicolumn{1}{c}{意味} \\\hline
ENAME     & Invalid file name           & ファイル名が不正 \\
ENOENT    & No such file or directrory  & ファイルが存在しない \\
EEXIST    & File exists                 & 同名ファイルが存在する \\
EOPEND    & File is opened              & 既にオープンされている \\
ENFILE    & File table overflow         & システムのオープン超過 \\
EBADF     & Bad file number             & ファイル記述子が不正 \\
ENOSPC    & No space left on device     & デバイスに空き領域が不足 \\
EPATH     & Bad path                    & パスが不正 \\
EMODE     & Bad mode                    & モードが一致しない \\
EFATTR    & Bad attribute               & ファイルの属性が不正 \\
ENOTEMP   & Directory is not empty      & ディレクトリが空でない \\
EINVAL    & Invalid argument            & 引数が不正 \\
EMPROC    & Process table overflow      & プロセスが多すぎる \\
ENOEXEC   & Bad EXE file                & EXE ファイルが不正 \\
EMAGIC    & Bad MAGIC number            & 不正なマジック番号 \\
EMFILE    & Too many open files         & プロセス毎のオープン超過 \\
ECHILD    & No children                 & 子プロセスが存在しない \\
ENOZOMBI  & No zombie children          & ゾンビの子が存在しない \\
ENOMEM    & Not enough memory           & 十分な空き領域が無い \\
ESYSNUM   & Invalid system call number  & システムコール番号が不正 \\
EZERODIV  & Zero division               & ゼロ割り算 \\
EPRIVVIO  & Privilege violation         & 特権違反 \\
EILLINST  & Illegal instruction         & 不正命令 \\
EUSTK     & Stack overflow              & スタックオーバーフロー \\
EUMODE    & stdio: Bad open mode        & モードと使用方法が矛盾 \\
EUBADF    & stdio: Bad file pointer     & 不正な fp が使用された \\
EUEOF     & fgetc: EOF was ignored      & fgetc前にEOFチェック必要 \\
EUNFILE   & fopen: Too many open files  & プロセス毎のオープン超過 \\
EUSTDIO   & fclose: Standard i/o should & 標準ioはクローズできない \\
          &  not be closed              &                          \\
EUFMT     & fprintf: Invalid conversion & 書式文字列に不正な変換 \\
EUNOMEM   & malloc: Insufficient memory & ヒープ領域が不足 \\
EUBADA    & free: Bad address           & mallocした領域ではない \\
\end{tabular}
\end{mytable}

\section{標準関数}

\verb/#include <stdlib.hmm>/を書いた後で使用します。

\subsection{malloc関数}

ヒープ領域に引数の整数で指定バイト数のメモリ領域を確保し、
領域を指す参照を返します。
\verb/malloc/関数は\verb/void[]/型なので、
領域を指す参照は全ての参照変数に代入できます。

\tac 版では、ヒープ領域に十分な空きが見つからないとき
終了コード\verb/EUNOMEM/でプログラムを終了します。
トランスレータ版では、エラーメッセージを表示したあと
終了コード\verb/1/でプログラムを終了します。

\begin{quote}
\begin{verbatim}
public void[] malloc(int size);
\end{verbatim}
\end{quote}

\subsection{free関数}

\verb/malloc/関数で割当てた領域を解放します。
\tac 版では、領域が\verb/malloc/関数で割当てたものではない可能性がある場合
(マジックナンバーが破壊されている、管理されている空き領域と重なる等)、
終了コード\verb/EUBADA/でプログラムを終了します。

\begin{quote}
\begin{verbatim}
public void free(void[] mem);
\end{verbatim}
\end{quote}

\subsection{atoi関数}

\verb/atoi/関数は引数に渡した10進数文字列を解析して、
それが表現する値を返します。

\begin{quote}
\begin{verbatim}
public int atoi(char[] s);
\end{verbatim}
\end{quote}

\subsection{srand関数}

\verb/srand/関数は擬似乱数発生器を\verb/seed/で初期化します。

\begin{quote}
\begin{verbatim}
public void srand(int seed);
\end{verbatim}
\end{quote}

\subsection{rand関数}

\verb/rand/関数は次の擬似乱数を発生します。

\begin{quote}
\begin{verbatim}
public int rand();
\end{verbatim}
\end{quote}

\subsection{exit関数}

\verb/exit/関数はオープン済みのストリームをフラッシュしてから
プログラムを終了します。

\verb/status/は、親プロセスに返す終了コードです。
\verb/0/が正常終了の意味、\verb/1/以上はユーザが決めた終了コード、
負の値は\tabref{chap4:err}に示す記号名で定義されています。
負の値を返すと親プロセスがシェルの場合、
シェル側でエラーメッセージを表示してくれます。

\begin{quote}
\begin{verbatim}
public void exit(int status);
\end{verbatim}
\end{quote}

\section{文字列操作関数}

\verb/#include <string.hmm>/を書いた後で使用します。

\subsection{strCpy関数}

\verb/s/文字列を文字配列\verb/d/にコピーし、
\verb/d/を関数値として返します。
トランスレータ版では\cl の\verb/strcpy/関数に置き換えられます。

\begin{quote}
\begin{verbatim}
public char[] strCpy(char[] d, char[] s);
\end{verbatim}
\end{quote}

\subsection{strNcpy関数}

文字列\verb/s/の最大\verb/n/文字を文字配列\verb/d/にコピーし、
\verb/d/を関数値として返します。
文字配列の使用されない部分には\verb/'\0'/が書き込まれます。
文字列の長さが\verb/n-1/より長い場合は、
\verb/'\0'/が書き込まれないので注意して下さい。
トランスレータ版では\cl の\verb/strncpy/関数に置き換えられます。

\begin{quote}
\begin{verbatim}
public char[] strNcpy(char[] d, char[] s, int n);
\end{verbatim}
\end{quote}

\subsection{strCat関数}

文字列\verb/s/を文字配列\verb/d/に格納されている文字列の後ろに追加し、
\verb/d/を関数値として返します。
トランスレータ版では\cl の\verb/strcat/関数に置き換えられます。

\begin{quote}
\begin{verbatim}
public char[] strCat(char[] d, char[] s);
\end{verbatim}
\end{quote}

\subsection{strNcat関数}

文字列\verb/s/の先頭\verb/n/文字未満を、
文字配列\verb/d/に格納されている文字列の後ろに追加し、
\verb/d/を関数値として返します。
\verb/d/に格納された文字列の最後には\verb/'\0'/が書き込まれます。
トランスレータ版では\cl の\verb/stnrcat/関数に置き換えられます。

\begin{quote}
\begin{verbatim}
public char[] strNcat(char[] d, char[] s, int n);
\end{verbatim}
\end{quote}

\subsection{strCmp関数}

文字列\verb/s1/と文字列\verb/s2/を比較します。
\verb/strCmp/関数は、アスキーコード順で\verb/s1/が大きいとき正の値、
\verb/s1/が小さいとき負の値、同じ時\verb/0/を返します。
トランスレータ版では\cl の\verb/strcmp/関数に置き換えられます。

\begin{quote}
\begin{verbatim}
public int strCmp(char[] d, char[] s);
\end{verbatim}
\end{quote}

\subsection{strNcmp関数}

文字列\verb/s1/と文字列\verb/s2/の先頭\verb/n/文字を比較します。
\verb/strNcmp/関数は、
\verb/strcmp/関数同様にアスキーコード順で大小を判断します。
トランスレータ版では\cl の\verb/strncmp/関数に置き換えられます。

\begin{quote}
\begin{verbatim}
public int strNcmp(char[] d, char[] s, int n);
\end{verbatim}
\end{quote}

\subsection{strLen関数}

文字列\verb/s/の長さを返します。
長さに\verb/'\0'/は含まれません。
トランスレータ版では\cl の\verb/strlen/関数に置き換えられます。

\begin{quote}
\begin{verbatim}
public int strLen(char[] s);
\end{verbatim}
\end{quote}

\subsection{strChr関数}

文字列\verb/s/の中で最初に文字\verb/c/が現れる位置を、
\verb/s/文字配列の添字で返します。
文字\verb/c/が含まれていない場合は\verb/-1/を返します。
トランスレータ版では\verb;lib/wrapper.c;の関数に置き換えられます。

\begin{quote}
\begin{verbatim}
public int strChr(char[] s, char c);
\end{verbatim}
\end{quote}

\subsection{strRchr関数}

文字列\verb/s/の中で最後に文字\verb/c/が現れる位置を、
\verb/s/文字配列の添字で返します。
文字\verb/c/が含まれていない場合は\verb/-1/を返します。
トランスレータ版では\verb;lib/wrapper.c;の関数に置き換えられます。

\begin{quote}
\begin{verbatim}
public int strRchr(char[] s, char c);
\end{verbatim}
\end{quote}

\subsection{strStr関数}

文字列\verb/s1/の中に文字列\verb/s2/が現れる位置を、
\verb/s1/文字配列の添字で返します。
文字列\verb/s2/が含まれていない場合は\verb/-1/を返します。
トランスレータ版では\verb;lib/wrapper.c;の関数に置き換えられます。

\begin{quote}
\begin{verbatim}
public int strStr(char[] s1, char[] s2);
\end{verbatim}
\end{quote}

\subsection{subStr関数}

文字列\verb/s/の先頭\verb/pos/文字を省いた部分文字列を返します。
返した文字列は\verb/s/とメモリ領域を共用していますので注意が必要です。
トランスレータ版では\verb;lib/wrapper.c;の関数に置き換えられます。

\begin{quote}
\begin{verbatim}
public char[] subStr(char[] s, int pos);
\end{verbatim}
\end{quote}

\section{参照型(アドレス)操作関数}

\cmml にはキャスティング演算や、ポインタ演算がありません。
そこで、以下の関数で代用します。

\subsection{\ul \ul ItoA 関数}
\label{chap4:itoa}

整数から参照へ型を変換する関数です。
16ビット整数を引数に \verb/void[]/ 参照(16ビットのアドレス)を返します。
\ul \ul ItoA 関数のプロトタイプ宣言を次に示します。
関数の値は \verb/void[]/ 型の参照なので、
どのような参照型変数にも代入できます。

\begin{quote}
\begin{verbatim}
public void[] __ItoA(int a);
\end{verbatim}
\end{quote}

\ul \ul ItoA 関数の応用例を次に示します。
\pageref{app:memmap}ページに示すように
F000H 番地は \tac のビデオメモリの開始番地です。
このプログラムは、
ビデオメモリを配列のようにアクセスし画面を文字 'A' で埋めつくします。

\begin{mylist}
\begin{verbatim}
// ビデオ RAM を配列としてアクセスする
char[] vRam = __ItoA(0xf000);
for (int i=0; i<80*24; i=i+1)
  vRam[i] = 'A';
\end{verbatim}
\end{mylist}

\subsection{\ul \ul AtoI 関数}

参照から整数へ型を変換する関数です。
参照(16ビットアドレス)を引数に16ビットの整数を返します。
\ul \ul AtoI 関数のプロトタイプ宣言を次に示します。
引数の型は \verb/void[]/ なので、
参照型ならどんな型でも渡すことができます。

\begin{quote}
\begin{verbatim}
public int __AtoI(void[] a);
\end{verbatim}
\end{quote}

\ul \ul AtoI 関数の応用例を次に示します。
このプログラムは、\tac の IPL ルーチン(実物)の一部です。
まず、指定セクタから読んだブートプログラムをメモリに格納します。
次に、このプログラムの先頭アドレスにジャンプします。

\begin{mylist}
\begin{verbatim}
// IPL の一部(ブートプログラムを読み込む)
  char[] hBuf = malloc(256);          // マイクロドライブ用バッファ
  int    sct = 23;                    // 23セクタから格納されている
  readSct(0,sct,hBuf);                // 1セクタ読み出す
  char[] mem = __ItoA(bswap(hBuf[0]));// 先頭ワードがロードアドレス
  int    len = bswap(hBuf[1]);        // 第2ワードがプログラム長
  int    idx = 2;                     // bswapは上/下位バイトを交換
  int    offs = 0;
  while(true) {
     while (idx<256) {                // 1セクタ256ワードに付き
       mem[offs] = bswap(hBuf[idx]);  // 機械語をメモリにコピー
       offs = offs + 1;
       idx  = idx  + 1;
     }
     if (offs>=len) break;            // プログラム長を超えたら終了
     sct = sct + 1;                   // 次のセクタを読む
     readSct(0,sct,hBuf);
     idx = 0;
  }
  putstr("\n\nStart boot loader@0x");
  puthex(__AtoI(mem));
  putstr(" ...\n\n");
  _jump(__AtoI(mem));                 // ロードしたプログラムに飛ぶ
\end{verbatim}
\end{mylist}

\subsection{\ul \ul AddrAdd 関数}

\cl のポインタ演算の代用にする関数です。
参照(アドレス)と整数を引数に渡し、
参照から整数ワード先の参照(アドレス)を返します。
この関数は \verb/malloc/、\verb/free/ 関数の実現に使用されています。
\ul \ul AddrAdd 関数のプロトタイプ宣言を次に示します。

\begin{quote}
\begin{verbatim}
public void[] __AddrAdd(void[] a, int n);
\end{verbatim}
\end{quote}

\subsection{\ul \ul acmp 関数}

参照(アドレス)の大小比較を行う関数です。
\cmml では参照の大小比較はできません。
\javal でも参照の大小比較はできないので、通常はこの仕様で十分と考えられます。
しかし、\verb/malloc/、\verb/free/ 関数等の実現には
アドレスの大小比較が必要です。
そこで、アドレスの大小比較をする \ul \ul acmp 関数を用意しました。
\ul \ul acmp 関数のプロトタイプ宣言は下のようになります。
\ul \ul acmp 関数は、\verb/a/ の方が大きい場合は 1 を、
\verb/b/ の方が大きい場合は -1 を、
\verb/a/ と \verb/b/ が等しい場合は 0 を返します。

\begin{quote}
\begin{verbatim}
public int __acmp(void[] a, void[] b);
\end{verbatim}
\end{quote}

\section{符号無し整数の操作関数}

\cmml では、int 型も char 型も符号付き整数です。
内部表現に2の補数表現を用いているので、
ほとんどの操作は、
符号付きと考えても符号無しと考えても同じ機械語命令列で処理できます。
そのため、符号付き数を符号無し数の代用に使用することができます。
しかし、大小比較は符号付きと符号無しで異なる機械語を必要とします。
そこで、この部分だけ関数を使用した特別な比較方法を使います。

\subsection{\ul \ul ucmp 関数}

符号無し数の比較を行う関数です。
\ul \ul ucmp 関数のプロトタイプ宣言を次に示します。
\ul \ul ucmp 関数は、\verb/a/ の方が大きい場合は 1 を、
\verb/b/ の方が大きい場合は -1 を、
\verb/a/ と \verb/b/ が等しい場合は 0 を返します。

\begin{quote}
\begin{verbatim}
public int __ucmp(int a, int b);
\end{verbatim}
\end{quote}

次に \ul \ul ucmp 関数の応用例を示します。
二つの符号無し数の大小関係によって3通りに場合分けしています。

\begin{mylist}
\begin{verbatim}
  int a = 0xffff;
  int b = 0x0000;
  if (__ucmp(a,b)>0) {
    ...                          // a > b の場合
  } else if (__ucmp(a,b)==0) {
    ...                          // a = b の場合
  } else if (__ucmp(a,b)<0) {
    ...                          // a < b の場合
  }
\end{verbatim}
\end{mylist}

\section{可変個引数関数のサポート関数}

\cmml は引数の数が一定ではない可変個引数関数の宣言を認めますが、
当該関数を記述するために必要な可変個引数へのアクセス方法を提供しません。
そこで、サポート関数の力を借りて引数へアクセスする手段を提供します。

\subsection{\ul \ul args 関数}
\label{chap4:args}

printf 関数のような可変個引数の関数を実現するために、
可変個引数関数の内部で引数を配列としてアクセスできるようにする関数です。
\ul \ul args 関数は \ul \ul args を呼び出した
\cmm 関数の第2引数を添字 0 とする配列を返します。
\ul \ul args 関数のプロトタイプ宣言を次に示します。

\begin{quote}
\begin{verbatim}
public int[] __args();
\end{verbatim}
\end{quote}

次に可変個引数関数の記述例として、
printf 関数を記述したプログラムの一部を示します。

\begin{mylist}
\begin{verbatim}
int printf(char[] fmt, ...) {      // ... は可変個引数の関数を表す
  int[] args = __args();           // args配列は第2引数以降を格納
  int n = 0;                       // 引数用のカウンタ
  int arg = args[n];               // 第2引数から順に取り出す
  for (i=0; fmt[i]!='\0'; i=i+1) { // 第1引数fmtは普通に使用できる
    char c = fmt[i];               // 書式文字列から1文字取り出す
    if (c=='%') {                  // 書式文字列に % 発見
      c = fmt[i=i+1];              // % に続く文字を取り出す
      ...
      } else if (c=='c') {         // %c なら
        putch(arg);                // 1文字出力
        arg = args[n=n+1];         // arg を更新
      } else if (c=='s') {         // %s なら
        putstr(__ItoA(arg));       // 引数は文字配列参照のはず
        arg = args[n=n+1];         // arg を更新
      } else ...
    ...
\end{verbatim}
\end{mylist}


\section{32ビット演算関数}

\cmml には32ビット整数型がありません。
32ビットの四則演算をするための関数を用意しました。
これらの関数では32ビットデータを要素数2の int 型配列で表現します。

\subsection{\ul add32 関数}

32ビットの加算ルーチンです。
dst、src 二つの配列を渡し結果を dst 配列に格納します。
また、関数値として dst 配列への参照を返します。
つぎに、\ul add32 関数のプロトタイプ宣言を示します。

\begin{quote}
\begin{verbatim}
public int[] _add32(int[] dst, int[] src);
\end{verbatim}
\end{quote}

\ul add32 関数の使用例を次に示します。
%最後の行は、\ul add32 関数が dst 配列への参照を返すことを利用して、
%直前の2行と同じ処理をコンパクトに記述した例になっています。

\begin{mylist}
\begin{verbatim}
int[] one = { 0, 1 };               // 32ビットの 1
int[] two = { 0, 2 };               // 32ビットの 2
int[] dat = array(2);

dat[0] = dataH;                     // 32ビットデータ
dat[1] = dataL;

_add32(dat, one);                   // データに +1 する
_add32(dat, two);                   // データに +2 する

_add32(_add32(dat, one), two);      // 前の2行と同じ演算を短く記述

printf("%04x%04x\n",dat[0],dat[1]); // 計算結果表示
\end{verbatim}
\end{mylist}

\subsection{\ul sub32 関数}

32ビットの減算ルーチンです。
dst、src 二つの配列を渡し dst から src を引いた結果を dst 配列に格納します。
また、関数値として dst 配列への参照を返します。
つぎに、\ul sub32 関数のプロトタイプ宣言を示します。

\begin{quote}
\begin{verbatim}
public int[] _sub32(int[] dst, int[] src);
\end{verbatim}
\end{quote}

\subsection{\ul mul32 関数}

32ビットの乗算ルーチンです。
16ビットデータと16ビットデータの積を32ビットデータとして返します。
一方の16ビットデータを dst 配列の2番目の要素(\verb/dst[1]/)に格納して、
もう一方の16ビットデータを src として渡します。
dst 配列の1番目の要素(\verb/dst[0]/)は計算に使用しません。
計算結果は dst 配列に32ビットデータとして格納します。
また、関数値として dst 配列への参照を返します。
つぎに、\ul mul32 関数のプロトタイプ宣言を示します。

\begin{quote}
\begin{verbatim}
public int[] _mul32(int[] dst, int src);
\end{verbatim}
\end{quote}

\subsection{\ul div32 関数}

32ビットの割算ルーチンです。
32ビットデータを16ビットデータで割って、
商と余りをそれぞれ16ビットで求めます。
32ビットデータを dst 配列で、
16ビットデータを src として渡します。
計算結果は、商が \verb/dst[0]/ に、余りが \verb/dst[1]/ に、
それぞれ16ビットデータとして格納されます。
また、関数値として dst 配列への参照を返します。
つぎに、\ul div32 関数のプロトタイプ宣言を示します。

\begin{quote}
\begin{verbatim}
public int[] _div32(int[] dst, int src);
\end{verbatim}
\end{quote}

\subsection{Ld32 マクロ}

32ビットデータを初期化するためのマクロです。
\verb/util.h/ ファイル中で次のように宣言されています。

\begin{quote}
\begin{verbatim}
#define Ld32(dst,h,l) ((dst)[0]=(h),(dst)[1]=(l))
\end{verbatim}
\end{quote}

使用例を示します。

\begin{mylist}
\begin{verbatim}
int[] dat = array(2);

Ld32(dat, dataH, dataL);            // 32ビットデータ

_add32(dat, one);                   // データに +1 する
\end{verbatim}
\end{mylist}

\section{バイトアクセス関数}

Boot プログラム中でマイクロドライブの構造を解析する際に
必要になった関数です。
PC 等が書き込んだマイクロドライブ上のデータを TaC が読み込むと、
\cmml の配列に次表のような順序で格納されます。
このデータをバイト単位でアクセスできると便利です。

\begin{center}
{\bf マイクロドライブのデータを int 配列に格納した状態} \\
\begin{tabular}{| r | c | c |}
\hline
\multicolumn{1}{|c|}{添字} & \multicolumn{2}{|c|}{内 容}          \\
\hline
0 & 第2バイト & 第1バイト \\
\hline
1 & 第4バイト & 第3バイト \\
\hline
2 & 第6バイト & 第5バイト \\
\hline
      &       &       \\
\multicolumn{3}{|c|}{...}           \\
      &       &       \\
\hline
\end{tabular}
\end{center}

\subsection{byte 関数}

byte 関数は、
int 型配列 b に上記表の順にバイトデータが格納されているとみなし、
第 x バイトを取り出します。
第1バイトを取り出したいときは、x を 0 にします。
次に byte 関数のプロトタイプ宣言を示します。

\begin{quote}
\begin{verbatim}
public int byte(int[] b, int x); 
\end{verbatim}
\end{quote}

\subsection{Word マクロ}

バイト配列からバイト単位で位置を指定して、
16ビットデータを取り出すマクロです。
\verb/util.h/ ファイル中で次のように宣言されています。

\begin{quote}
\begin{verbatim}
#define Word(buf, idx) ((byte((buf),(idx)+1)<<8) | byte((buf),(idx)))
\end{verbatim}
\end{quote}
